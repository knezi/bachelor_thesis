%%% Hlavní soubor. Zde se definují základní parametry a odkazuje se na ostatní části. %%%

%% Verze pro jednostranný tisk:
% Okraje: levý 40mm, pravý 25mm, horní a dolní 25mm
% (ale pozor, LaTeX si sám přidává 1in)
\documentclass[12pt,a4paper]{report}
\setlength\textwidth{145mm}
\setlength\textheight{247mm}
\setlength\oddsidemargin{15mm}
\setlength\evensidemargin{15mm}
\setlength\topmargin{0mm}
\setlength\headsep{0mm}
\setlength\headheight{0mm}
% \openright zařídí, aby následující text začínal na pravé straně knihy
\let\openright=\clearpage

%% Pokud tiskneme oboustranně:
% \documentclass[12pt,a4paper,twoside,openright]{report}
% \setlength\textwidth{145mm}
% \setlength\textheight{247mm}
% \setlength\oddsidemargin{14.2mm}
% \setlength\evensidemargin{0mm}
% \setlength\topmargin{0mm}
% \setlength\headsep{0mm}
% \setlength\headheight{0mm}
% \let\openright=\cleardoublepage

%% Vytváříme PDF/A-2u
\usepackage[a-2u]{pdfx}

%% Přepneme na českou sazbu a fonty Latin Modern
\usepackage[czech]{babel}
\usepackage{lmodern}
\usepackage[T1]{fontenc}
\usepackage{textcomp}

%% Použité kódování znaků: obvykle latin2, cp1250 nebo utf8:
\usepackage[utf8]{inputenc}

%%% Další užitečné balíčky (jsou součástí běžných distribucí LaTeXu)
\usepackage{amsmath}        % rozšíření pro sazbu matematiky
\usepackage{amsfonts}       % matematické fonty
\usepackage{amsthm}         % sazba vět, definic apod.
\usepackage{bbding}         % balíček s nejrůznějšími symboly
			    % (čtverečky, hvězdičky, tužtičky, nůžtičky, ...)
\usepackage{bm}             % tučné symboly (příkaz \bm)
\usepackage{graphicx}       % vkládání obrázků
\usepackage{fancyvrb}       % vylepšené prostředí pro strojové písmo
\usepackage{indentfirst}    % zavede odsazení 1. odstavce kapitoly
\usepackage{natbib}         % zajištuje možnost odkazovat na literaturu
			    % stylem AUTOR (ROK), resp. AUTOR [ČÍSLO]
\usepackage[nottoc]{tocbibind} % zajistí přidání seznamu literatury,
                            % obrázků a tabulek do obsahu
\usepackage{icomma}         % inteligetní čárka v matematickém módu
\usepackage{dcolumn}        % lepší zarovnání sloupců v tabulkách
\usepackage{booktabs}       % lepší vodorovné linky v tabulkách
\usepackage{paralist}       % lepší enumerate a itemize
\usepackage[usenames]{xcolor}  % barevná sazba

%% SPECIMEN
% Takto označená část slouží pro tvorbu vzorového PDF vystaveného na webu.
% Během generování oficiální šablony skriptem ../mkdist se automaticky smaže,
% stejně jako všechna volání maker \X a \XXX.
\def\X#1{\textcolor{red}{[#1]}}
\def\XXX#1{\par\smallskip\noindent \textcolor{red}{[#1]}}
%% NEMICEPS

%%% Údaje o práci

% Název práce v jazyce práce (přesně podle zadání)
\def\NazevPrace{Název práce \X{přesně podle zadání}}

% Název práce v angličtině
\def\NazevPraceEN{Name of thesis \X{přesný překlad do angličtiny}}

% Jméno autora
\def\AutorPrace{Jméno Příjmení}

% Rok odevzdání
\def\RokOdevzdani{ROK}

% Název katedry nebo ústavu, kde byla práce oficiálně zadána
% (dle Organizační struktury MFF UK, případně plný název pracoviště mimo MFF)
\def\Katedra{Název katedry nebo ústavu \X{dle organizační struktury MFF UK}}
\def\KatedraEN{Name of the department}

% Jedná se o katedru (department) nebo o ústav (institute)?
\def\TypPracoviste{Katedra}
\def\TypPracovisteEN{Department}

% Vedoucí práce: Jméno a příjmení s~tituly
\def\Vedouci{Vedoucí práce \X{s~tituly}}

% Pracoviště vedoucího (opět dle Organizační struktury MFF)
\def\KatedraVedouciho{katedra}
\def\KatedraVedoucihoEN{department}

% Studijní program a obor
\def\StudijniProgram{studijní program}
\def\StudijniObor{studijní obor}

% Nepovinné poděkování (vedoucímu práce, konzultantovi, tomu, kdo
% zapůjčil software, literaturu apod.)
\def\Podekovani{%
Poděkování.
}

% Abstrakt (doporučený rozsah cca 80-200 slov; nejedná se o zadání práce)
\def\Abstrakt{%
Abstrakt. \X{doporučeno cca 80--200 slov; nejedná se o~zadání práce}
}
\def\AbstraktEN{%
Abstract.
}

% 3 až 5 klíčových slov (doporučeno), každé uzavřeno ve složených závorkách
\def\KlicovaSlova{%
{klíčová} {slova} \X{obvykle 3 až~5 klíčových slov či frází}
}
\def\KlicovaSlovaEN{%
{key} {words}
}

%% Balíček hyperref, kterým jdou vyrábět klikací odkazy v PDF,
%% ale hlavně ho používáme k uložení metadat do PDF (včetně obsahu).
%% Většinu nastavítek přednastaví balíček pdfx.
\hypersetup{unicode}
\hypersetup{breaklinks=true}

%% Definice různých užitečných maker (viz popis uvnitř souboru)
\include{makra}

%% Titulní strana a různé povinné informační strany
\begin{document}
\include{titulka}

%%% Strana s automaticky generovaným obsahem bakalářské práce

\tableofcontents

%%% Jednotlivé kapitoly práce jsou pro přehlednost uloženy v samostatných souborech
\include{uvod}
\include{kap01}
\include{kap02}
\include{kap03}
\include{kap04}

\include{zaver}

%%% Seznam použité literatury
\include{literatura}

%%% Obrázky v bakalářské práci
%%% (pokud jich je malé množství, obvykle není třeba seznam uvádět)
\listoffigures

%%% Tabulky v bakalářské práci (opět nemusí být nutné uvádět)
%%% U matematických prací může být lepší přemístit seznam tabulek na začátek práce.
\listoftables
\XXX{U~matematických prací může být lepší přemístit seznam tabulek na začátek práce.}

%%% Použité zkratky v bakalářské práci (opět nemusí být nutné uvádět)
%%% U matematických prací může být lepší přemístit seznam zkratek na začátek práce.
\chapwithtoc{Seznam použitých zkratek}
\XXX{U~matematických prací může být lepší přemístit seznam zkratek na začátek práce.}

%% PHDONLY
%%% Součástí doktorských prací musí být seznam vlastních publikací
\chapwithtoc{Seznam publikací}
\XXX{Součástí doktorských prací musí být seznam vlastních publikací.}
%% ONLYPHD

%%% Přílohy k bakalářské práci, existují-li. Každá příloha musí být alespoň jednou
%%% odkazována z vlastního textu práce. Přílohy se číslují.
%%%
%%% Do tištěné verze se spíše hodí přílohy, které lze číst a prohlížet (dodatečné
%%% tabulky a grafy, různé textové doplňky, ukázky výstupů z počítačových programů,
%%% apod.). Do elektronické verze se hodí přílohy, které budou spíše používány
%%% v elektronické podobě než čteny (zdrojové kódy programů, datové soubory,
%%% interaktivní grafy apod.). Elektronické přílohy se nahrávají do SISu a lze
%%% je také do práce vložit na CD/DVD. Povolené formáty souborů specifikuje
%%% opatření rektora č. 72/2017.
\appendix
\chapter{Přílohy}
\XXX{Přílohy k bakalářské práci, existují-li. Každá příloha musí být alespoň jednou odkazována z vlastního textu práce. Přílohy se číslují.}
\XXX{Do tištěné verze se spíše hodí přílohy, které lze číst a prohlížet (dodatečné tabulky a grafy, různé textové doplňky, ukázky výstupů z počítačových programů, apod.). Do elektronické verze se hodí přílohy, které budou spíše používány v~elektronické podobě než čteny (zdrojové kódy programů, datové soubory, interaktivní grafy apod.). Elektronické přílohy se nahrávají do SISu a lze je také do práce vložit na CD/DVD. Povolené formáty souborů specifikuje opatření rektora č.~72/2017.}

\section{První příloha}

\openright
\end{document}
