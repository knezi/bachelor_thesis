\chapter*{Appendix A}
\label{app:a}
\addcontentsline{toc}{chapter}{Appendix A}


\section{Format of the Dataset}
\label{sec:format}

The Yelp Dataset contains various files.
We will use files \texttt{review.json} and \texttt{business.json}.
The former contains reviews.
Review is represented by JSON and there is one review per line\footnote{In fact, it is JSON lines format --- \url{https://jsonlines.org}}.
Every review contains unique ID, user ID, business ID which is a reference to a business in file \texttt{business.json}.
It also contains data about the review itself; review text, stars, date of publishing and number of individual likes it received so far.
An example of pretified review is in \todoA{ref to reivew}.

\begin{code}
{
	"review_id": "---nya_pjxWmNFDFyAcfsA",
	"user_id": "5QOtcHU1SoqEqBCRR6FhsA",
	"business_id": "zQNJwaWR1M1zDjLNVJiNEw",
	"stars": 1,
	"date": "2012-06-27",
	"text": "Another case of the Emperor's New Clothes...",
	"useful": 10,
	"funny": 2,
	"cool": 3
}
\end{code}

The second file we use is \texttt{business.json}.
Again, it is in JSON lines and it contains business information such as openning hours or availability of a car park.
An example of a business can be found bellow.

\begin{code}
{
	"business_id": "zQNJwaWR1M1zDjLNVJiNEw",
	"name": "Pizza M",
	"neighborhood": "",
	"address": "208 W Main St",
	"city": "Urbana",
	"state": "IL",
	"postal_code": "61801",
	"latitude": 40.112655,
	"longitude": -88.2093142,
	"stars": 3.5,
	"review_count": 60,
	"is_open": 1,
	"attributes": {
		"RestaurantsTableService": false,
		"Alcohol": "beer_and_wine",
		"Caters": false,
		"HasTV": false,
		"RestaurantsGoodForGroups": true,
		"NoiseLevel": "average",
		"WiFi": "free",
		"RestaurantsAttire": "casual",
		"RestaurantsReservations": false,
		"OutdoorSeating": false,
		"BusinessAcceptsCreditCards": true,
		"RestaurantsPriceRange2": 2,
		"BikeParking": true,
		"RestaurantsDelivery": true,
		"RestaurantsTakeOut": true,
		"GoodForKids": true,
		"BusinessParking": {
			"garage": true,
			"street": true,
			"validated": false,
			"lot": false,
			"valet": false
		}
	},
	"categories": ["Restaurants", "Pizza"],
	"hours": {
		"Monday": "11:00-21:00",
		"Tuesday": "11:00-21:00",
		"Friday": "11:00-22:00",
		"Wednesday": "11:00-22:00",
		"Thursday": "11:00-22:00",
		"Sunday": "11:00-21:00",
		"Saturday": "11:00-23:00"
	}
}
\end{code}

\section{Data Preprocessing}

This thesis focuses on text classification and therefore we did not want to use unnecessary data.
At the end, we used business information only for checking the type of business and whether its reviews are reliable.
We use only reviews belonging to businesses with at least 50 reviews and 10 attributes.
This is to omit infrequently visited restaurants where the number of likes could fluctuate a lot.

We also wanted to focus on English text.
Besides English, two other languages (German and French) were frequently present.
We decided to filter out the two languages and keep other languages, because they presence is not significant.
We tried very simple approach.
Because we already use a spellchecker\footnote{see bellow}, we tried to use the ratio of incorrect words to detect language.
We computed the ratio of misspelled and correct words for English, German and French and picked the language with the lowest ratio.
It worked surprisingly well, even though many English words such as ``credit card'' or ``hamburger'' are used in the other languages too.

Note that we focus on restaurants only in this thesis, but in the dataset there are general businesses.
However, we extracted about 206,000 reviews out of which about 177,000 are restaurants.
Because the number of non-restaurants is so low, we decided to treat all businesses the same way as restaurants.

For easier manipulation, we denormalize the data and add information from spellchecker.
We create an instance file which is in JSON lines containing an instance per line.
An instance is a review with the element \texttt{business\_id} replaced by the business information.
Also, we ran a spellchecker\footnote{url{https://aspell.net}} on every review and added the information of number of words and number of misspelled words to the review.
An example of a full instance is bellow.

\begin{code}
{
	"review_id": "---nya_pjxWmNFDFyAcfsA",
	"user_id": "5QOtcHU1SoqEqBCRR6FhsA",
	"business_id": {
		"business_id": "zQNJwaWR1M1zDjLNVJiNEw",
		"name": "Pizza M",
		"neighborhood": "",
		"address": "208 W Main St",
		"city": "Urbana",
		"state": "IL",
		"postal_code": "61801",
		"latitude": 40.112655,
		"longitude": -88.2093142,
		"stars": 3.5,
		"review_count": 60,
		"is_open": 1,
		"attributes": {
			"RestaurantsTableService": false,
			"Alcohol": "beer_and_wine",
			"Caters": false,
			"HasTV": false,
			"RestaurantsGoodForGroups": true,
			"NoiseLevel": "average",
			"WiFi": "free",
			"RestaurantsAttire": "casual",
			"RestaurantsReservations": false,
			"OutdoorSeating": false,
			"BusinessAcceptsCreditCards": true,
			"RestaurantsPriceRange2": 2,
			"BikeParking": true,
			"RestaurantsDelivery": true,
			"RestaurantsTakeOut": true,
			"GoodForKids": true,
			"BusinessParking": {
				"garage": true,
				"street": true,
				"validated": false,
				"lot": false,
				"valet": false
			}
		},
		"categories": ["Restaurants", "Pizza"],
		"hours": {
			"Monday": "11:00-21:00",
			"Tuesday": "11:00-21:00",
			"Friday": "11:00-22:00",
			"Wednesday": "11:00-22:00",
			"Thursday": "11:00-22:00",
			"Sunday": "11:00-21:00",
			"Saturday": "11:00-23:00"
		}
	},
	"stars": 1,
	"date": "2012-06-27",
	"text": "Another case of the Emperor's New Clothes...",
	"useful": 10,
	"funny": 2,
	"cool": 3,
	"words": 57,
	"incorrect_words": 0
}
\end{code}

\todoA{add spellcheck mention - lang recognisition}


TODO lanuage


\section{Geneea Data}




\section{Special Files}

this probably should be moved to architecture

In \autoref{chap:arch}, we refer to special files.
Let us describe their exact format here.

\begin{table}[h]

\centering
\begin{tabular}{ll}
\toprule
\textbf{name of file}& \textbf{purpose} \\
\midrule
instance file		 & stores instances in the final format \\
intermediate file	 & used for generating Geneea file \\
Geneea file			 & contains extra linguistics data for instances \\
experiments file	 & configures the experiments conducted \\
\bottomrule
\end{tabular}

\caption{File definition}\label{tab:files}
\end{table}
There sbe on line-by-line correspondence between the Geneea file and the data file prepared locally.

\todoA{either filter out non-resta, or say it's negligible}.  
