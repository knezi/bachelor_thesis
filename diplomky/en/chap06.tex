\chapter{Experiments Conducted}\label{chap:exp}

In this chapter, we outline all experiments conducted and discuss the results.
\todoA{az to bude hotove}

Since the objective of this thesis is to compare several approaches,
next step after building the architecture was to define sets of experiments and metrics to compare the results.
Our original intention was to try all combinations of features, classifiers and other techniques.
However, this turned out to be computationally infeasible.
Instead, we tried promising subsets by only choosing the best variant out of each group and combine it.

The training and testing was done on a dataset with 64,550 instances.
For each evaluation, 10-fold crossvalidation was used, resulting in
training sets with 58,095 instances and testing with 6,455 instances.

For labels, we used the number of useful likes.
Reviews with zero likes were labelled not-useful and
reviews with at least two likes useful.
1 like reviews were dropped.

We split features into the sets defined in the two lists bellow.
The first list contains existential features --- they express that certain phrase is present in the review.

\begin{itemize}
	\item UNIGRAMS --- top 50,000 unigrams according to mutual information
	\item BIGRAMS --- bigrams consisting of top 25,000 adjacent unigrams
	\item TRIGRAMS  --- trigrams consisting of top 10,000 adjacent unigrams
	\item TFIDF  --- top 50,000 words
	\item ENTITIES --- all 270,00 entities as found by the Geneea analyzer
\end{itemize}

The threshold of n-grams and TF-IDF were found experimentally to produce approximately the same number of features as entities.
\todoA{not true}

The second list contains threshold features --- they split reviews into groups based on some property reaching a threshold.

\begin{itemize}
	\item STARS 
		\begin{itemize}
			\item the number of stars; five groups
			\item indicators for each stars number; five pairs one-vs-rest
			\item extreme stars; one group 1 or 5 stars; the rest another group
			\item the number of average stars for the business; five groups
		\end{itemize}
	\item REVIEWLEN 
		\begin{itemize}
			\item the number of words; three groups with thresholds 50 and 150
			\item the number of words; five pairs with thresholds 35, 50, 75, 100 and 150
		\end{itemize}
	\item SPELLCHECK 
		\begin{itemize}
			\item the rate of misspelled words; five pairs with thresholds 0.02, 0.05, 0.1, 0.15 and 0.2
			\item the number of misspelled words; four pairs with thresholds 5, 10, 15 and 20
		\end{itemize}
	
	\item COSINESIM --- cosine similarity to 10 randomly chosen useful instances; five pairs with thresholds 0.4, 0.6, 0.8, 0.9, 0.95
\end{itemize}

We performed the analysis in five rounds.
We report f-measure and accuracy of all models to retain comparability even throughout rounds.
Each round focuses on a different variable as listed bellow:

\begin{enumerate}
	\item baseline; (plus compute mutual information)
	\item feature sets with Na\"{i}ve Bayes
	\item feature selection algorithms
	\item classifiers
	\item size of training data
\end{enumerate}


\section{1st round --- Baseline}

In this round, we ran zero-R and one-R to get an idea about the problem.
Also, we computed mutual information of all features.

Zero-R has accuracy


\section{2nd round --- Features}

In this round, we run Na\"{i}ve Bayes with different sets of features.
Configurations with results are shown in \autoref{tab:feat_perf}.
Names refer to feature sets.
\{uni,bi\}grams and \{uni,bi,tri\}grams are listed n-grams together.
All\_basic is STARS, REVIEWLEN, SPELLCHECK and COSINESIM.
Plus sign denotes a feature set combined with all\_basic.

\begin{table}[h!]

\centering
\begin{tabular}{lr@{~}r@{~}rr@{~}r@{~}r}
\toprule
\textbf{name}	& \multicolumn{3}{c}{\textbf{f-measure}} & \multicolumn{3}{c}{\textbf{accuracy}} \\
\midrule
unigrams& 0.47 & (0.46, 0.48) & $\pm$ 0.007 & 0.3 & (0.29, 0.32) & $\pm$ 0.006 \\
tfidf& 0.47 & (0.46, 0.48) & $\pm$ 0.007 & 0.3 & (0.29, 0.32) & $\pm$ 0.006 \\
\{uni,bi\}grams	& 0.47 & (0.46, 0.47) & $\pm$ 0.005 & 0.3 & (0.3, 0.31) & $\pm$ 0.004\\
bigrams & 0.47 & (0.46, 0.47) & $\pm$ 0.005 & 0.31 & (0.3, 0.31) & $\pm$ 0.005			\\
entities & 0.48 & (0.47, 0.49) & $\pm$ 0.005 & 0.35 & (0.34, 0.36) & $\pm$ 0.005		\\
\{uni,bi,tri\}grams & 0.47 & (0.46, 0.48) & $\pm$ 0.008 & 0.3 & (0.29, 0.32) & $\pm$ 0.007	\\
trigrams & 0.47 & (0.46, 0.49) & $\pm$ 0.008 & 0.32 & (0.31, 0.33) & $\pm$ 0.006		\\
cosine\_sim & -0.4 & (-1.0, 0.003) & $\pm$ 0.52 & 0.7 & (0.69, 0.7) & $\pm$ 0.006		\\
all\_basic & 0.59 & (0.58, 0.6) & $\pm$ 0.005 & 0.71 & (0.71, 0.72) & $\pm$ 0.004		\\
bigrams+ & 0.48 & (0.47, 0.49) & $\pm$ 0.007 & 0.35 & (0.34, 0.36) & $\pm$ 0.006		\\
tfidf+ & 0.48 & (0.47, 0.5) & $\pm$ 0.007 & 0.36 & (0.35, 0.37) & $\pm$ 0.006			\\
entities+ & 0.55 & (0.54, 0.56) & $\pm$ 0.008 & 0.54 & (0.53, 0.55) & $\pm$ 0.006		\\
\bottomrule
\end{tabular}






\caption{Performance of Feature Configurations}\label{tab:feat_perf}
All reported values are in the format value (min, max) $\pm$ standard deviation.
\end{table}


\section{3rd round --- Feature Selection}

In this round, we compare three different methods of feature selection to performance when all features are used as are.
We use bigrams+ from the previous round.
Na\"{i}ve Bayes with BIGRAMS, STARS, REVIEWLEN, COSINESIM and SPELLCHECK from the previous round is used.
We run PCA with 100 dimensions as recommended by scipy documentation.
For mutual information and chi square, we choose top 1000 features.

\begin{table}[h!]

\centering
\begin{tabular}{lllll}
\toprule
\textbf{name}\\	%& \textbf{f-measure} & \textbf{accuracy} \\
\midrule
bigrams+ \\
bigrams+[MI] \\
bigrams+[CHI] \\
bigrams+[PCA] \\
\bottomrule
\end{tabular}

\caption{Performance of Feature Configurations}\label{tab:feat_perf}
\end{table}

\section{4th round --- Classifiers}

\section{5th round --- Training Size}

We use processed data as described in \autoref{app:a}.

also mention what filter condition has been used, and geenea


\label{chap:exp}

\todoB[try them all - possible discussion on difficulty on different types].

All measurements will be with respect to the~size of dataset, to see how the~usefulness of approaches improves with more data.

criteria (TD-idf, entropy...), then try to add *-grams. Later, try more linguistics features.

In the~next part, if you have time, try other approaches to improve performance. You can try PCA reduction, NN, or other ML algorithms/techniques.

* experiments conducted + results (what data, what features, expectations)

\todoA{promysli ficury desc}

