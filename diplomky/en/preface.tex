\chapter*{Introduction}
\addcontentsline{toc}{chapter}{Introduction}

With growing popularity of storing and sharing more data on the internet, the biggest challenge
for us is not to get enough information, but to find the exact information we need. This thesis tries to tackle
one of the major issues of review systems. Usually, the review systems have many reviews and it is
hard to filter those that are useful, namely because we do not know the author (hence cannot estimate
his or her reliability), but also because reviews are very short, often containing just one piece of information.
Hence it is difficult for users to decide what to trust and what not.

We will use data set containt reviews of restaurants and try to 
filter out only useful reviews. To a lessen degree we will also filter reviews that have been labelled by users as ``funny'' and ``cool''.
This filtering can be later used as part of the review system where the platform automatically provides users with useful reviews first.

Our main goal is to compare already existing approaches and to find the best one for the given circumstances.

\todoA{call it labelling instead of classificaotn?}


\todoA{conventions}


{\bf definitons}

{\it names of algorithms, approaches}

\texttt{files, code}

\todoA{list of chapters what is where}
