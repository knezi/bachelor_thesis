\chapter*{Introduction}
\addcontentsline{toc}{chapter}{Introduction}

With growing popularity of storing and sharing more data on the internet, the biggest challenge
for us is not to get enough information, but to find the exact information we need.
This thesis tries to tackle one of the major issues of review systems.
A review system is a platform for sharing experience among its users.
It is particularly aimed at information that is hard to find in another way than from direct experience.
The information is often subjective and relevant only in a narrow context.
Review of a novel listing historical accuracy as an advantage may be interesting to a history teacher,
but not so much to someone who only searches for a gripping story.

In particular, users use review systems for obtaining some information based on the experience of others.
The users are often willing to spend only a couple of minutes on finding relevant information to them.
Usually, the review systems have many reviews and review systems tries to improve the user experience by showing useful reviews first.

However, it is not obvious how to find useful reviews.
First, we do not know the author and possibly have only a few reviews by the autor.
Also, reviews by the same author can have different quality which is influenced by many factors,
such as time spent on writing the review or wheter the actual experience contains some useful information.
Therefore it is very hard to estimate author's reliability.
Second, reviews are very short, often containing only one piece of information.
Third, a review can quickly become obsolete or irrelevant.
Fourth, what is relevant to some, may not be to others.
All this also makes the task hard even for human operators and as such hard to evaluate our programme selecting useful reviews.

Also, the form and language of reviews is very specific.
\citet{Song14} define it as a new genre short text.
Short text is used in e-commerce systems and online communication in general.
It is usually up to 200 characters per document.
It is characterizes by its sparsity, free form and being large-scale and in realtime.
Although an average review in our data has 526 characters, we consider this short text.
First, it fullfills other characteristics.
Second, the bounds are not clearly defined.
An example of another short text system is Instant Messaging software Windows Live Messenger allowing up to 400 characters.

In this thesis, we try to address the issue of saying whether a review is useful.
We focus on data set containing reviews of restaurants.
Since it is hard to evaluate usefulness of a review, we use data where reviews can be manually marked by users as useful.
We use this information to assess how well our programme works.
To a lessen degree, we select reviews that have been marked as ``funny'' and ``cool''.

We use machine learning methods to solve this problem.
To date, many approches to solve this have been introduced.
However, there is no clearly best performing algorithm for this context.
Our main goal is to compare already existing approaches and compare them.
The results can serve as a guide when implementing a real review system.



\todoA{conventions}


{\bf definitons}

{\it names of algorithms, approaches}

\texttt{files, code}

\todoA{list of chapters what is where}

\todoA{add some sources doing the same}

\section{Related Work}

A lot of research has been done in short text classification.
Executive summary can be found in \citet{Song14}.
They define short text as a new genre; listing its characteristics, summarizing possible approaches to short text classification and evaluating performance.

One of the well studied data is tweets on Twitter like in \citep{sriram2010short}.
Reviews are to author's knowledge less studied.
One of the relevant papers is \citet{ganu2009beyond}.
They discuss the difficulties of selecting useful restaurant reviews.
Unlike this thesis, they do not classify reviews as a whole, but instead try to find reviews containing topics such as food or service.
The main attempt is to find some structure in the free form.

Although there is a lot of research focusing on sentiment analysis \todoA{some citations?},
very little research focuses on usefulness.
Relatively close research focuses on toxic speech.
\citet{van2018challenges} talk about toxic speech as non-constructive and non-inclusive conversation including hate speech, threats, insults and in general fruitless conversation.
Filtering out fruitless text is our focus to high extent too.
Apart from already mentioned challenges they mention high varience in performance and difficulty to define the topic clearly.
Basic review of toxic speech classification can be found in \citet{gunasekara2018review}.
