\chapter*{Conclusion}\label{chap:concl}
\addcontentsline{toc}{chapter}{Conclusion}

In this thesis, we focused on comparing various classification methods.
Our incentive was to filter useful reviews for a recommendation system,
in order to improve the user experience and usability.
For used the Yelp dataset to evaluate our results.

We have built an extensive software package for comparing classification models.
This package can be easily used in further research in machine learning 
as it allows extensibility and flexibility.

For the actual experiments, we tried different sets of features, feature selection algorithms and
classifiers.
Finally, we performed learning curves analysis to confirm the generalisability and look for
ways for further improvement.

The experiments showed it is a hard problem 
and that all our solutions were only insignificantly better than the baseline solution.
It has showed
we would need to study this problem more in depth to be able to gain a significant improvement.

Especially, the following areas, which are well beyond the scope of this thesis, could be extended:

\begin{itemize}

\item analysis of thresholds ---
most thresholds were chosen based on some previous research without adjusting them to our situation, insufficient analysis or simply our intuition.
In an ideal case, an optimization algorithm would be used to find the perfect values.

\item feature selection methods ---
except for TF-IDF, features were filtered by feature selection algorithms as a black box.
It may be beneficial to tune the selection exactly to our case or even try some wrappers or embedded methods.

\item dataset preparation ---
the data contain a lot of noise which may be possible to filter out.
Also, it is not very clear whether our labelling of the dataset is a reliable method.

\item classifier tuning ---
we used only three different classifiers with the default parameters.
More thorough tuning by using an optimization algorithm could be beneficial.

\item use more advanced techniques ---
our models may be too simple.
It may beneficial to use more sophisticated methods
such as ensemble classifiers or neural networks
 
\end{itemize}
